% Options for packages loaded elsewhere
\PassOptionsToPackage{unicode}{hyperref}
\PassOptionsToPackage{hyphens}{url}
%
\documentclass[
]{article}
\usepackage{amsmath,amssymb}
\usepackage{lmodern}
\usepackage{iftex}
\ifPDFTeX
  \usepackage[T1]{fontenc}
  \usepackage[utf8]{inputenc}
  \usepackage{textcomp} % provide euro and other symbols
\else % if luatex or xetex
  \usepackage{unicode-math}
  \defaultfontfeatures{Scale=MatchLowercase}
  \defaultfontfeatures[\rmfamily]{Ligatures=TeX,Scale=1}
\fi
% Use upquote if available, for straight quotes in verbatim environments
\IfFileExists{upquote.sty}{\usepackage{upquote}}{}
\IfFileExists{microtype.sty}{% use microtype if available
  \usepackage[]{microtype}
  \UseMicrotypeSet[protrusion]{basicmath} % disable protrusion for tt fonts
}{}
\makeatletter
\@ifundefined{KOMAClassName}{% if non-KOMA class
  \IfFileExists{parskip.sty}{%
    \usepackage{parskip}
  }{% else
    \setlength{\parindent}{0pt}
    \setlength{\parskip}{6pt plus 2pt minus 1pt}}
}{% if KOMA class
  \KOMAoptions{parskip=half}}
\makeatother
\usepackage{xcolor}
\IfFileExists{xurl.sty}{\usepackage{xurl}}{} % add URL line breaks if available
\IfFileExists{bookmark.sty}{\usepackage{bookmark}}{\usepackage{hyperref}}
\hypersetup{
  hidelinks,
  pdfcreator={LaTeX via pandoc}}
\urlstyle{same} % disable monospaced font for URLs
\usepackage[margin=1in]{geometry}
\usepackage{graphicx}
\makeatletter
\def\maxwidth{\ifdim\Gin@nat@width>\linewidth\linewidth\else\Gin@nat@width\fi}
\def\maxheight{\ifdim\Gin@nat@height>\textheight\textheight\else\Gin@nat@height\fi}
\makeatother
% Scale images if necessary, so that they will not overflow the page
% margins by default, and it is still possible to overwrite the defaults
% using explicit options in \includegraphics[width, height, ...]{}
\setkeys{Gin}{width=\maxwidth,height=\maxheight,keepaspectratio}
% Set default figure placement to htbp
\makeatletter
\def\fps@figure{htbp}
\makeatother
\setlength{\emergencystretch}{3em} % prevent overfull lines
\providecommand{\tightlist}{%
  \setlength{\itemsep}{0pt}\setlength{\parskip}{0pt}}
\setcounter{secnumdepth}{-\maxdimen} % remove section numbering
\usepackage[french, USenglish]{babel}
\usepackage{fancyhdr}
\pagestyle{fancy}
\renewcommand{\sectionmark}[1]{\markright{#1}}
\fancyhf{}
\lhead{{}}
\rhead{{}}
\cfoot{{\thepage}}
\usepackage[T1]{fontenc}
\usepackage{bm}
\usepackage{mathpazo}
\usepackage{tabularx}
\usepackage{titlesec}
\usepackage{graphicx, xcolor}
\usepackage{wrapfig}
\usepackage{amssymb}
\usepackage{amsmath}
\usepackage{lscape}
\usepackage{esint}
\usepackage{paralist}
\usepackage{outlines}
\newcommand{\I}{\textrm{I}}
\newcommand{\N}{\mathcal{N}}
\newcommand{\D}{\textrm{D}}
\newcommand{\E}{\mathbb{E}}
\setlength{\parskip}{1em} %0.5\baselineskip
\setlength{\parindent}{0pt}
\linespread{1.15}
\titleformat*{\section}{\Large\scshape\bfseries}
\titleformat*{\subsection}{\large\scshape\bfseries}
\titleformat*{\subsubsection}{\bfseries}
\titleformat*{\paragraph}{\bfseries}
\titleformat*{\subparagraph}{\bfseries}
\renewcommand{\thesection}{\Roman{section}.} % 1.A. as subsections
\renewcommand{\thesubsection}{\Alph{subsection}.} % 1.A. as subsections
\titlespacing{\section}{0pt}{2pt}{3pt}
\titlespacing{\subsection}{0pt}{2pt}{2pt}
\titlespacing{\subsubsection}{0pt}{0pt}{0pt}
\titlespacing{\paragraph}{0pt}{1pt}{5pt}
\titlespacing{\subparagraph}{10pt}{1pt}{5pt}
\usepackage{hyperref}
\hypersetup{ colorlinks=true, citecolor = blue, linkcolor=blue, urlcolor=blue}
\usepackage[font={footnotesize}]{subcaption}
\usepackage[font={footnotesize}]{caption}
\usepackage{caption, setspace}
\captionsetup{font={stretch=1}}
\captionsetup[figure]{font=footnotesize,labelfont=footnotesize}
\usepackage{tabto}
\def\quoteattr#1#2{\setbox0=\hbox{#2}#1\tabto{\dimexpr\linewidth-\wd0}\box0}
\makeatletter
\newcommand{\pushright}[1]{\ifmeasuring@#1\hfill$\displaystyle#1$\fi\ignorespaces}
\makeatother
\newcommand{\FixMe}[1]{\textcolor{orange}{ [#1]}}
\newcommand{\Comment}[1]{\textcolor{purple}{\textit{[#1]}}}
\newcommand{\Quickwin}{{\color{blue}{$\bigstar$}} }
\renewcommand{\thetable}{S\arabic{table}}
\renewcommand{\thefigure}{S\arabic{figure}}
\usepackage{letltxmacro}
\LetLtxMacro\Oldfootnote\footnote
\newcommand{\EnableFootNotes}{\LetLtxMacro\footnote\Oldfootnote}
\newcommand{\DisableFootNotes}{\renewcommand{\footnote}[2][]{\relax}}
\makeatother
\usepackage{lscape}
\newcommand{\blandscape}{\begin{landscape}}
\newcommand{\elandscape}{\end{landscape}}
\graphicspath{{../Output/"}}
\usepackage{booktabs}
\usepackage{longtable}
\usepackage{array}
\usepackage{multirow}
\usepackage{wrapfig}
\usepackage{float}
\usepackage{colortbl}
\usepackage{pdflscape}
\usepackage{tabu}
\usepackage{threeparttable}
\usepackage{threeparttablex}
\usepackage[normalem]{ulem}
\usepackage{makecell}
\usepackage{xcolor}
\ifLuaTeX
  \usepackage{selnolig}  % disable illegal ligatures
\fi

\author{}
\date{\vspace{-2.5em}}

\begin{document}

\begin{center} 
    \textbf{\scshape \LARGE Supplemental Information}\\  \vspace{2mm}
    {\large COVID-19 and Excess All-Cause Mortality in the US and 20 Comparison Countries, \\ June 2021-March 2022}\\ \vspace{2mm} 
{\large Alyssa Bilinski\footnote{Contact: alyssa\_bilinski@brown.edu} $\cdot$ Kathryn Thompson $\cdot$ Ezekiel Emanuel} \\
\end{center}

\vspace{-0.50em}

\bigskip
\bigskip

\hypertarget{data-sources}{%
\subsection{Data sources}\label{data-sources}}

We compared the US overall and the 10 most- and least-vaccinated states
to 20 Organization for Economic Co-operation and Development (OECD)
countries with 2021 population exceeding 5 million
(\href{https://stats.oecd.org/Index.aspx?DataSetCode=HISTPOP}{link}) and
greater than \$25,000 per capita gross domestic product in 2021
(\href{https://data.worldbank.org/indicator/NY.GDP.PCAP.PP.CD}{link}).

For the US, where we required sub-national data for our analyses, we
obtained data from CDC files: COVID-19 mortality from ``United States
COVID-19 Cases and Deaths by State over Time''
(\href{https://healthdata.gov/dataset/United-States-COVID-19-Cases-and-Deaths-by-State-o/hiyb-zgc2}{link}),
all-cause mortality data from ``Weekly Counts of Deaths by Jurisdiction
and Age''
(\href{https://data.cdc.gov/NCHS/Weekly-Counts-of-Deaths-by-Jurisdiction-and-Age/y5bj-9g5w}{link}),
and vaccination data from ``COVID-19 Vaccination Trends in the United
States, National and Jurisdictional''
(\href{https://data.cdc.gov/Vaccinations/COVID-19-Vaccination-Trends-in-the-United-States-N/rh2h-3yt2}{link}).
US population data were obtained from the Census
(\href{https://www.census.gov/data/tables/time-series/demo/popest/2010s-state-total.html}{link},
\href{https://www.census.gov/data/tables/time-series/demo/popest/2020s-state-total.html}{link}).

For other countries, we obtained data on COVID-19 mortality from the
World Health Organization (\href{https://covid19.who.int/data}{link}),
all-cause mortality estimates from OECD.Stat
(\href{https://stats.oecd.org/Index.aspx?QueryId=96018}{link}), and
vaccination data from Our World In Data, which aggregates local
estimates (\href{https://ourworldindata.org/covid-vaccinations}{link}).
We checked data sources by matching US CDC data to WHO and OECD
mortality estimates and spot-checking OECD all-cause mortality against
country-specific estimates.

COVID-19 mortality was reported daily, and we aggregated weeks beginning
on Sunday (``CDC'' or ``epi weeks''). All-cause mortality was reported
weekly. In the US, new weeks began on Sunday (``epi weeks''), but in
other countries, weekly data began on Monday (``ISO weeks''). We defined
delta and omicron periods based on visual inspection of mortality
trends, during summer 2021 for delta and December 2022 for omicron.
Because waves were tightly clustered in time, we defined the start of
periods based on the earliest mortality turning point for delta and
omicron across all locations of interest. Code and additional analyses
are available on
\href{https://github.com/abilinski/ExcessMortalityDeltaOmicron}{GitHub}.

\bigskip

\hypertarget{calculations-in-tables-1-and-2}{%
\subsection{Calculations in Tables 1 and
2}\label{calculations-in-tables-1-and-2}}

\hypertarget{potential-us-deaths-averted}{%
\subsubsection{Potential US deaths
averted}\label{potential-us-deaths-averted}}

Let \(r_i\) be the death rate of interest (reported COVID-19 deaths or
excess all cause mortality) per 100,000 in country \(i\), and \(d\) be
US deaths over the period of interest. Let \(p\) be the US population in
the year of interest. We estimate the difference in deaths or potential
US deaths averted: \begin{align*}
d - \left(r_i/100,000\right)*p
\end{align*}

\newpage

\hypertarget{states-by-vaccination-rate}{%
\subsection{States by Vaccination
Rate}\label{states-by-vaccination-rate}}

\begin{table}[H]

\caption{\label{tab:unnamed-chunk-1}States by 2-dose vaccination status (as of January 1, 2022). Nine of the 10 most-vaccinated states and 8 of the 10 least-vaccinated states were consistent over the full study period from June 2021 through March 2022; results were robust to including only these states in the top and bottom 10 over the full period (omitting New York from the top 10 and Indiana and North Dakota from the bottom 10) (see \href{https://github.com/abilinski/ExcessMortalityDeltaOmicron}{GitHub}).}
\centering
\begin{tabular}[t]{l|r|r|l}
\hline
State & Rate & Rank & Set\\
\hline
\cellcolor{gray!6}{Vermont} & \cellcolor{gray!6}{79.0} & \cellcolor{gray!6}{1} & \cellcolor{gray!6}{Top 10}\\
\hline
Rhode Island & 77.6 & 2 & Top 10\\
\hline
\cellcolor{gray!6}{Maine} & \cellcolor{gray!6}{77.0} & \cellcolor{gray!6}{3} & \cellcolor{gray!6}{Top 10}\\
\hline
Connecticut & 75.5 & 4 & Top 10\\
\hline
\cellcolor{gray!6}{Hawaii} & \cellcolor{gray!6}{75.3} & \cellcolor{gray!6}{5} & \cellcolor{gray!6}{Top 10}\\
\hline
Massachusetts & 75.1 & 6 & Top 10\\
\hline
\cellcolor{gray!6}{New York} & \cellcolor{gray!6}{72.7} & \cellcolor{gray!6}{7} & \cellcolor{gray!6}{Top 10}\\
\hline
Maryland & 72.0 & 8 & Top 10\\
\hline
\cellcolor{gray!6}{New Jersey} & \cellcolor{gray!6}{71.8} & \cellcolor{gray!6}{9} & \cellcolor{gray!6}{Top 10}\\
\hline
District of Columbia & 71.3 & 10 & Top 10\\
\hline
\cellcolor{gray!6}{Indiana} & \cellcolor{gray!6}{53.4} & \cellcolor{gray!6}{42} & \cellcolor{gray!6}{Bottom 10}\\
\hline
North Dakota & 53.4 & 42 & Bottom 10\\
\hline
\cellcolor{gray!6}{Tennessee} & \cellcolor{gray!6}{52.4} & \cellcolor{gray!6}{44} & \cellcolor{gray!6}{Bottom 10}\\
\hline
Arkansas & 52.3 & 45 & Bottom 10\\
\hline
\cellcolor{gray!6}{Georgia} & \cellcolor{gray!6}{52.2} & \cellcolor{gray!6}{46} & \cellcolor{gray!6}{Bottom 10}\\
\hline
Idaho & 52.0 & 47 & Bottom 10\\
\hline
\cellcolor{gray!6}{Louisiana} & \cellcolor{gray!6}{51.1} & \cellcolor{gray!6}{48} & \cellcolor{gray!6}{Bottom 10}\\
\hline
Mississippi & 49.6 & 49 & Bottom 10\\
\hline
\cellcolor{gray!6}{Wyoming} & \cellcolor{gray!6}{49.1} & \cellcolor{gray!6}{50} & \cellcolor{gray!6}{Bottom 10}\\
\hline
Alabama & 48.9 & 51 & Bottom 10\\
\hline
\end{tabular}
\end{table}

\hypertarget{excess-mortality-model-selection-table-2}{%
\subsection{Excess Mortality Model Selection (Table
2)}\label{excess-mortality-model-selection-table-2}}

We used the following procedure to select models for estimating excess
mortality, designed to address concerns about appropriately fitting
secular trends raised in prior work
(\href{https://www.nber.org/papers/w29503}{link}). In response to this
issue, some authors have fit parametric pre-pandemic trends (e.g.,
\href{https://www.nber.org/papers/w29503}{link},
\href{https://www.nber.org/papers/w30104}{link}) while others have used
only 2019 as a benchmark year for counterfactual non-pandemic mortality
(e.g., \href{https://www.nber.org/papers/w30512}{link}). To choose
functional form, we evaluated 3 possible models on pre-pandemic data: 1)
a model fit only on the most recent year of data, 2) models with
country-specific trends as fixed effects, and 3) models with
country-specific trends random effects. \begin{align}
y_{ctk} &= {W}_{ct} + \epsilon_{ctk} &\text{fit to only the most-recent year of data}\\
y_{ctk} &= {W}_{ct} + \beta_c^F k + \epsilon_{ctk} \\
y_{ctk} &= {W}_{ct} + \beta_c^R k + \epsilon_{ctk} 
\end{align}

where \(y_{ctk}\) was mortality per 100,000 population in week \(t\) of
year \(k\) in location \(c\), \({W}_{ct}\) was a week-location fixed
effect, \(k\) indicated year, and \(\epsilon_{ctk}\) was residual error.
Country-specific linear trend parameters \(\beta_c^F\) were estimated as
fixed effects while \(\beta_c^R\) were estimated as random effects. We
considered both models fit on all data and models fit separately on the
set of weeks in each period (delta: 26-51 and omicron: 52-53, 1-12).

We first used 2018 and 2019 as test data. We fit models (1)-(3) on data
from 2015-2017, predicting out-of-sample mortality for each country-week
in 2018, and 2015-2018, predicting out-of-sample mortality in 2019. We
estimated root mean-squared error for each period of weeks (delta: 26-51
and omicron: 52-53, 1-12) (\(\mathcal{P}\)) and test year (K) at the
week-level
\(\left(RMSE_{\mathcal{P}, K} = \sqrt{\frac{1}{N_c} \sum_{c \in \mathcal{C}} \sum_{t \in \mathcal{P}} \left(y_{ctK}-\hat{y}_{ctK}\right)^2}\right)\)
and at the period-level
\(\left(RMSE_{\mathcal{P}, K}^{CP} = \sqrt{\frac{1}{N_c} \sum_{c \in \mathcal{C}} \left(\sum_{t \in \mathcal{P}} \left(y_{ctK}-\hat{y}_{ctK}\right)\right)^2}\right)\).
We also ranked models by RMSE within each country and evaluated mean
model ranks. We found that model (3) (random effects), estimated with
all weeks of data (rather than separately for each period), either
strictly or weakly dominated all other models over these metrics during
these test years and periods.

Second, we fit models for data from 2015-2017, and predicted 2019 to
approximate 2021 with only 2019 pre-intervention data. (We did not want
to fit trend models on only 2 years of pre-pandemic data and therefore
did not use 2018 as a test year.) Model (3) continued to outperform
other models overall; for \(RMSE^{CP}\) during omicron weeks, model (2)
estimated on all pre-intervention data slightly outperformed model (3),
but the magnitude of the difference was negligible. We therefore used
model (3) estimated over all weeks combined as our main specification.
See
\href{https://github.com/abilinski/ExcessMortalityDeltaOmicron}{GitHub}
for further notes, summaries of model evaluation statistics, and
sensitivity analyses.

\bigskip

\hypertarget{statistical-analyses}{%
\subsection{Statistical Analyses}\label{statistical-analyses}}

\hypertarget{table-1}{%
\subsubsection{Table 1}\label{table-1}}

To compare COVID-19 death rates across locations, we let \(d_{ct}\) be
the number of COVID-19 deaths in location \(c\) during period \(t\) and
\(p_{ct}\) be its population at time \(t\). We assumed that
\(d_{ct} \sim Pois(\lambda_{ct})\) and \begin{align*}
\mathbb{E}\left[log(\lambda_{ct})\right] = \beta_0 +  \sum_{j \in \mathcal{C}, j \neq US} \beta_j \mathbb{I}\left(c = j\right) + log(p_{ct}),
\end{align*} where \(\beta_j\) compared the death rate in country \(j\)
to the United States. We employed a similar model to make comparisons
with the 10 most-vaccinated states and 10 least-vaccinated states. We
used standard Wald tests to evaluate statistical significance.

\bigskip

\hypertarget{table-2}{%
\subsubsection{Table 2}\label{table-2}}

Per the model-fitting process described above, we modeled excess
all-cause mortality: \begin{align*}
y_{ctk} &= \sum_{j \in \mathcal{C}} \sum_{w \in \mathcal{W}} \sum_{\ell \geq 2020} \beta_{jw\ell} \mathbb{I}\{c=j \cap t = w \cap k = \ell\} + {W}_{ct} + \beta_c^R k + \epsilon_{ctk},
\end{align*} where \(y_{ctk}\) was mortality per 100,000 population in
week \(t\) of year \(k\) in location \(c\), \({W}_{ct}\) was a
location-week fixed effect, and \(k\) indicated year. We estimated
\(\beta_c^R\) as a random effect. Week-level treatment effects were
saturated to avoid estimating pre-pandemic trends with pandemic data. In
this framework, the linear combination of
\(\sum_{j = Comparator} \sum_{w,\ell \in \mathcal{P}} \beta_{j w \ell} - \sum_{j = US} \sum_{w, \ell \in \mathcal{P}} \beta_{jw\ell}\)
compared excess mortality between each comparator location and the US
over the set of week-years in period \(\mathcal{P}\). We used standard
Wald tests to evaluate statistical significance. We made similar
comparisons between the 10 most-vaccinated states and 10
least-vaccinated states, and also compared state subgroups to other
countries. To estimate the difference between COVID-19 and all-cause
mortality, we used as an outcome \(y_{ctk} - d_{ctk}\), where
\(d_{ctk}\) was COVID-19 mortality per 100,000 for location \(c\) in
week \(t\) and year \(k\).

\end{document}
